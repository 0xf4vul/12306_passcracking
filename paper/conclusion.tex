\section{Conclusion}
In this work, we conduct a comprehensive quantitative study on how
user personal information resides in human-chosen passwords. To the
best of our knowledge, we are the first to systematically analyze
personal information in passwords. We have some interesting and
quantitative discovery such as 3.42\% of the users in 12306 dataset
use their birthday as passwords, and male users are much more likely
to include their name in passwords than female users. We then develop
a novel metric, Coverage, that describes the correlation between a set
of personal information and passwords. Besides the natural belief that
longer personal information should yield higher Coverage, Coverage
also stresses that a continued personal information match is much
stronger than fragmented matches. We develop Personal-PCFG method that
which leverages the basic idea of PCFG but adds more semantic symbols
to crack passwords. Personal-PCFG generates personalized password
guesses by integrating user personal information in the guesses, so it
is significantly faster than PCFG and makes online attacks
feasible. Finally, we propose to use distortion functions to protect
passwords. Through a proof-of-concept study, we verify that distortion
functions are effective to defend personal information related and
semantic-aware attacks.
