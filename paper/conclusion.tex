\section{Conclusion}
\label{sec:conclusion}
In this work, we conduct a comprehensive quantitative study on how
user personal information resides in human-chosen passwords. To the
best of our knowledge, we are the first to systematically analyze
personal information in passwords. We have some interesting and
quantitative discovery such as 3.42\% of the users in the 12306 dataset
use their birthday as passwords, and male users are more likely
to include their name in passwords than female users. We then introduce
a new metric, Coverage, to accurately quantify the correlation between personal information and a password. 
%Besides the natural belief thatlonger personal information should yield higher Coverage, 
%Under the usage of Coverage,  a continuous match is much stronger than fragmented matches.
Our coverage-based quantification results further confirm our disclosure on the serious involvement of personal information in password creation, which makes a user password more vulnerable to a targeted password cracking.
We develop Personal-PCFG based on PCFG but consider more semantic symbols for cracking a password. Personal-PCFG generates personalized password guesses by integrating user personal information in the guesses. Our experimental results demonstrate that Personal-PCFG
is significantly faster than PCFG in password cracking and eases the feasibility of mounting online attacks.
Finally, we propose to use distortion functions to protect weak passwords that include personal information. Through a proof-of-concept study, we validate that distortion
functions are effective to defend against personal-information-related and semantics-aware attacks.
