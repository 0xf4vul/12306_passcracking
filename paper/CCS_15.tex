% This is "sig-alternate.tex" V2.0 May 2012
% This file should be compiled with V2.5 of "sig-alternate.cls" May 2012
%
% This example file demonstrates the use of the 'sig-alternate.cls'
% V2.5 LaTeX2e document class file. It is for those submitting
% articles to ACM Conference Proceedings WHO DO NOT WISH TO
% STRICTLY ADHERE TO THE SIGS (PUBS-BOARD-ENDORSED) STYLE.
% The 'sig-alternate.cls' file will produce a similar-looking,
% albeit, 'tighter' paper resulting in, invariably, fewer pages.
%
% ----------------------------------------------------------------------------------------------------------------
% This .tex file (and associated .cls V2.5) produces:
%       1) The Permission Statement
%       2) The Conference (location) Info information
%       3) The Copyright Line with ACM data
%       4) NO page numbers
%
% as against the acm_proc_article-sp.cls file which
% DOES NOT produce 1) thru' 3) above.
%
% Using 'sig-alternate.cls' you have control, however, from within
% the source .tex file, over both the CopyrightYear
% (defaulted to 200X) and the ACM Copyright Data
% (defaulted to X-XXXXX-XX-X/XX/XX).
% e.g.
% \CopyrightYear{2007} will cause 2007 to appear in the copyright line.
% \crdata{0-12345-67-8/90/12} will cause 0-12345-67-8/90/12 to appear in the copyright line.
%
% ---------------------------------------------------------------------------------------------------------------
% This .tex source is an example which *does* use
% the .bib file (from which the .bbl file % is produced).
% REMEMBER HOWEVER: After having produced the .bbl file,
% and prior to final submission, you *NEED* to 'insert'
% your .bbl file into your source .tex file so as to provide
% ONE 'self-contained' source file.
%
% ================= IF YOU HAVE QUESTIONS =======================
% Questions regarding the SIGS styles, SIGS policies and
% procedures, Conferences etc. should be sent to
% Adrienne Griscti (griscti@acm.org)
%
% Technical questions _only_ to
% Gerald Murray (murray@hq.acm.org)
% ===============================================================
%
% For tracking purposes - this is V2.0 - May 2012

\documentclass{sig-alternate}

\usepackage{booktabs,caption,fixltx2e}
\usepackage[flushleft]{threeparttable}
\usepackage{algorithm}
\usepackage{algpseudocode}
\usepackage{pifont}
\usepackage{setspace}
\usepackage{adjustbox}
\usepackage{graphicx}
\usepackage{multirow}
\begin{document}
%
% --- Author Metadata here ---
\conferenceinfo{WOODSTOCK}{'97 El Paso, Texas USA}
%\CopyrightYear{2007} % Allows default copyright year (20XX) to be over-ridden - IF NEED BE.
%\crdata{0-12345-67-8/90/01}  % Allows default copyright data (0-89791-88-6/97/05) to be over-ridden - IF NEED BE.
% --- End of Author Metadata ---

\title{How does personal information reside in human-chosen passwords? \\ --A quantitative study}

% You need the command \numberofauthors to handle the 'placement
% and alignment' of the authors beneath the title.
%
% For aesthetic reasons, we recommend 'three authors at a time'
% i.e. three 'name/affiliation blocks' be placed beneath the title.
%
% NOTE: You are NOT restricted in how many 'rows' of
% "name/affiliations" may appear. We just ask that you restrict
% the number of 'columns' to three.
%
% Because of the available 'opening page real-estate'
% we ask you to refrain from putting more than six authors
% (two rows with three columns) beneath the article title.
% More than six makes the first-page appear very cluttered indeed.
%
% Use the \alignauthor commands to handle the names
% and affiliations for an 'aesthetic maximum' of six authors.
% Add names, affiliations, addresses for
% the seventh etc. author(s) as the argument for the
% \additionalauthors command.
% These 'additional authors' will be output/set for you
% without further effort on your part as the last section in
% the body of your article BEFORE References or any Appendices.

\numberofauthors{3} %  in this sample file, there are a *total*
% of EIGHT authors. SIX appear on the 'first-page' (for formatting
% reasons) and the remaining two appear in the \additionalauthors section.
%
\author{
% You can go ahead and credit any number of authors here,
% e.g. one 'row of three' or two rows (consisting of one row of three
% and a second row of one, two or three).
%
% The command \alignauthor (no curly braces needed) should
% precede each author name, affiliation/snail-mail address and
% e-mail address. Additionally, tag each line of
% affiliation/address with \affaddr, and tag the
% e-mail address with \email.
%
% 1st. author
  % use '\and' if you need 'another row' of author names
% 4th. author
\alignauthor Yue Li\\
       \affaddr{College of William \& Mary}\\
       \email{yli@cs.wm.edu}
% 5th. author
\alignauthor Haining Wang\\
       \affaddr{University of Delaware}\\
       \email{hnw@udel.edu}
% 6th. author
\alignauthor Kun Sun\\
       \affaddr{College of William \& Mary}\\
       \email{ksun@cs.wm.com}
}
% There's nothing stopping you putting the seventh, eighth, etc.
% author on the opening page (as the 'third row') but we ask,
% for aesthetic reasons that you place these 'additional authors'
% in the \additional authors block, viz.
\additionalauthors{Additional authors: John Smith (The Th{\o}rv{\"a}ld Group,
email: {\texttt{jsmith@affiliation.org}}) and Julius P.~Kumquat
(The Kumquat Consortium, email: {\texttt{jpkumquat@consortium.net}}).}
\date{30 July 1999}
% Just remember to make sure that the TOTAL number of authors
% is the number that will appear on the first page PLUS the
% number that will appear in the \additionalauthors section.

\maketitle
\begin{abstract}
Left blank.
\end{abstract}

% A category with the (minimum) three required fields
\category{}{Security and privacy}{}[Human and societal aspects of security and privacy]
%A category including the fourth, optional field follows...
\category{}{General and reference}{}[Metrics]

\terms{Security}

\keywords{passwords, password cracking, data processing, password protection}

\section{Introduction}
Left blank

\section{Personal Info in Passwords}
Human-generated passwords are long criticized to be weak. Numerous works have shown that due to memorability requirement, users are more likely to use meaningful strings as their passwords. Therefore, passwords are usually very different from real random strings. For example, "password" is more likely a password than "ziorqpe". As a result, most passwords are within only a small portion of the large password space, making password guessing a lot easier. A natural question is: how do users choose their passwords so that they are different from random strings? The answer to the question has great significance for it has strong security implication to both users as well as systems. If an attacker knows exactly how users construct their passwords, cracking their passwords will become an easy task. On the other hand, if a user knows how other users construct their passwords, the user can easily improve his/her password strength by avoid using these password construction methods. 

To this end, researchers have done much to unveil the composition of passwords. Traditional dictionary attacks on passwords have shown that users tend to use dictionary words to construct their passwords. [][] claims that the distribution of characters in passwords is very similar to that in their native languages and people are prone to use words in their languages. [] Shows password words distribution is not the same as word distribution in the language. [Markov] shows that passwords are phonetically memorable. [PCFG] shows that using dictionary words to guess passwords is effective. [][][] indicate that users use keyboard strings such as "qwerty" and "qweasdzxc", trivial strings such as "password", "123456", and date strings such as "19951225" in their passwords.

As far as we see, most studies are done at a macro level. We now study user passwords in an individual base. We would like to show that user personal experience plays an important role when users create their passwords. Intuitively, people tend to choose their passwords based on their personal information because human beings are limited by their memory -- totally unrelated passwords are much less memorable.
 
\subsection{12306 Dataset}
In recent years, many password datasets are exposed to the public. Recent works on password measurement or password cracking are usually based on these datasets. Some of these datasets, such as Rockyou, are very large such that they even constitute millions of passwords. Now we are going to use a dataset which we call 12306 dataset to illustrate how personal information is used in user passwords.
\subsubsection{Introduction to dataset}
At the end of year 2014, a Chinese dataset is exposed to the public by anonymous attackers. It is said that the dataset is obtained using social engineering[], in which attackers use datasets at hand to try other websites. We call this dataset 12306 dataset because all passwords are from a website www.12306.com. The website is the official website for online railway ticket booking for Chinese users.

12306 dataset contains over 130,000 Chinese passwords. Having witnessed so many large datasets been leaked out, the size of 12306 dataset is just medium. What makes it special is that together with plain text passwords, the dataset also carries several types of user personal information. For example, user's name, ID number, etc. As the website needs real ID number to register and people need to provide real information to book a ticket, information in the dataset is considered reliable.

\subsubsection{Basic Measurement}
We do fundamental measurement to reveal some characteristics of 12306 dataset. After appropriate cleansing, we remove a minor part of passwords (0.2\%), with 131,389 good passowrds left for analysis. Note that websites may have different password creation policy. With strict password policy, users may apply mangling rules (For example, $abc -> @bc$ or $abc1$)to their passwords to fulfill the policy requirement. As 12306 website has changed its password policy after the password leakage, we do not know exactly the password policy at the time the dataset is leaked. However, from the dataset, we infer the password policy is quite simple -- all passwords need to be no shorter than 6 symbols. There is no restriction on what type of symbols are used. Therefore users are not forced to apply much mangling to their passwords. 

The average length of passwords in 12306 dataset is 8.44. Then we show the most common passwords in 12306 dataset. They are listed in Table~\ref{t1}.
\begin{table}
\centering
\caption{Most Frequent Passwords}
\begin{tabular}{|c|c|c|c|} \hline
Rank&Password&Amount&Percentage\\ \hline
1&123456&389&0.296\%\\ 
2&a123456&280&0.213\%\\ 
3&123456a&165&0.125\%\\ 
4&5201314&160&0.121\%\\ 
5&111111&156&0.118\%\\ 
6&woaini1314&134&0.101\%\\ 
7&qq123456&98&0.074\%\\ 
8&123123&97&0.073\%\\ 
9&000000&96&0.073\%\\ 
10&1qaz2wsx&92&0.070\%\\ 
\hline\end{tabular}
\label{t1}
\end{table}

From Table~\ref{t1} we can see that the dominating passwords are trivial passwords (123456, a123456, etc), keyboard passwords (1qaz2wsx and 1q2w3e4r), and "I love you" passwords. Both "5101314" and "woaini1314" means "I love you forever" in Chinese. The most commonly used Chinese passwords are similar to previous studies [fudan]. However, 12306 dataset is much less congregated. The most popular password "123456" accounts less than 0.3\% of all passwords while the number is 2.17\% in [Fudan]. We believe that the sparsity is due to the importance of the website so that users are less prone to use trivial passwords like "123456", etc. 

Then, we show the basic structure of passwords. The most popular password structures are shown in Table~\ref{t2}. Our result again shows that Chinese users prefer to use digits in their passwords instead of letters as in English-speaking users. The 5 top structures all have significant portion of digits, in which at most 2 or 3 letters are appended in front.

We reckon that the reason behind may be Chinese users lack vocabulary because Chinese use non-ASCII character set. Digits seem to be the best choice when creating a password.

\begin{table}
\centering
\caption{Most Frequent Password Structures}
\begin{tabular}{|c|c|c|c|} \hline
Rank&Structure&Amount&Percentage\\ \hline
1&$D_7$&10893&8.290\%\\ 
2&$D_8$&9442&7.186\%\\ 
3&$D_6$&9084&6.913\%\\ 
4&$L_2D_7$&5065&3.854\%\\ 
5&$L_3D_6$&4820&3.668\%\\ 
6&$L_1D_7$&4770&3.630\%\\ 
7&$L_2D_6$&4261&3.243\%\\ 
8&$L_3D_7$&3883&2.955\%\\ 
9&$D_9$&3590&2.732\%\\ 
10&$L_2D_8$&3362&2.558\%\\ 
\hline\end{tabular}
\begin{tablenotes}
      \small
      \item "D" represents digits and "L" represents English letters. The number indicates the segment length. For example, $D_7$ means the password contains 7 digits in a row.
    \end{tablenotes}
\label{t2}
\end{table}

In conclusion, 12306 dataset is a Chinese password dataset that has general Chinese password characteristics. However, its passwords are more sparse than previously studied datasets. 

\subsection{Personal Information}
As we have mentioned, 12306 dataset not only contains user passwords, it also carries multiple types of personal information. They are:

\begin{verbatim}
1. Name: User's Chinese name
2. Email address: User's registered email address
3. Cellphone number: User's registered cellphone number
4. Account name: the account used to log on the system, 
may contain digits and letters. For example, "myacct123".
5. ID number: Government issued ID number.
\end{verbatim}

Note that the government issued ID number is an 18-digit powerful number. These digits actually show personal information as well. Digit 1-6 represents the birth place of the owner, Digit 7-14 represents the birthday of the owner, and digit 17 represents the gender of the owner -- odd number means male and even number means female. We take out the 8-digit birthday information and treat it separately because birthday information is very important in a password. Therefore, we finally have 6 types personal information - 1)Name, 2)Birthday, 3)Email, 4)Cellphone 5)Account name, and 6) ID number (birthday not included). 

\subsubsection{New Password Representation}
To better illustrate how personal information correlates to user passwords, we develop a new representation of password which add more semantic symbols beside the conventional "D", "L" and "S" symbols, which means digit, letter, and special symbol accordingly. We try to match password to the 6 types of user personal information, and express the passwords with these personal information. For example, a password "alice1987abc" may be represented as $[Name][Birthday]L_3$ instead of $L_3D_4L_3$ in a traditional measurement. We substitute personal information with corresponding tag ([Name] and [Birthday] in this case). For the segments that are not matched, we still use "D","L", and "S" to describe the types of characters.

We believe representation like $[Name][Birthday]L_3$ is better than $L_5D_4L_3$ since it more accurately describe the composition of user passwords. We apply the matching to the whole 12306 dataset to see how these personal information tag appear in such password representations.

\subsubsection{Matching Method}
In order to make personal information password representations, an essential question will be: How do we match the personal information to user passwords? To answer this question, we show the algorithm we used in Algorithm~\ref{alg1}. The high level idea is that we find all substrings of the password and sort them in descending length order. Then we try to match the substrings from longest to shortest to all types of personal information. If one match is found, the leftover password segments are recursively applied the match function until no further match is found. Segments that are not matched to personal information will be processed using the traditional "LDS" method.


\begin{algorithm}[!h]
\caption{Match personal information with password}
\label{alg1}
\begin{algorithmic}[1]
\Procedure{Match}{$pwd$,$infolist$}
\State $newform \gets$ empty\_string
\If len($pwd$) == 0
\State \Return empty\_string
\EndIf
\State $substring \gets$ get\_all\_substring($pwd$)
\State reverse\_length\_sort($substring$)
\For {$eachstring$ \Pisymbol{psy}{206} $substring$}
\If {len($eachstring$) $\ge$ 2}
\If{matchbd($eachstring$,$infolist$)}
\State $tag \gets $ "[BD]"
\State $leftover \gets pwd$.split($eachstring$)
\State break
\EndIf
\State $\ldots$
\If{matchID($eachstring$,$infolist$)}
\State $tag \gets$ "[ID]"
\State $leftover \gets pwd$.split($eachstring$)
\State break
\EndIf
\Else
\State break
\EndIf
\EndFor
\If{$leftover$.size() $\ge$ 2}
\For{i $\gets$ 0 to $leftover$.size()-2}
\State $newform \gets$ MATCH($leftover[i]$,$infolist$) + $tag$
\EndFor
\State $newform \gets$  MATCH($leftover[leftover.size()-1]$)+$newform$
\Else
\State $newform \gets$ seg($pwd$)
\EndIf
\State \Return $newform$
\EndProcedure
\end{algorithmic}
\end{algorithm}

Note that we did not show specific matching algorithm to each type of the personal information (line 10 and line 16). To keep Algorithm~\ref{alg1} clean and simple, we describe the matching methods as follows.

First we make sure the password segments are at least length of 2 for matching. For segment of length 1, we directly map it to digit, letter, or special character. We try to match segments with length 2 or more to each kind of the information. For name information, we first convert Chinese names into Pinyin form, which is alphabetic representation of Chinese. Then we compare password segments to 10 possible permutations of the names, which include $lastname + firstname$, $last\_initial+firstname$, etc. If the segment is exactly same as any of the permutations, we consider a match is found. We list all the 10 permutations in the Appendices. For birthday information, we list 17 possible permutations and compare password segments to each of the permutation, if the segment is same as any permutations, we consider a match is found. We list all the birthday permutations in the Appendices. For account name, cellphone number, and ID number, we further restrain the length of segment to be at least 4 to avoid coincidence. We believe a match of length 4 is very likely to be an actual match. If the segment is a substring of any of the 3 personal information, we regard it a match to the corresponding personal information. 


\subsubsection{Matching Result}
After applying Algorithm~\ref{alg1} to 12306 dataset. We found that 71,037 out of 131,389 (54.1\%) of the passwords contain at least one of the 6 types of personal information. Apparently, personal information is an essential part of user passwords and most users put certain personal information in their passwords. We believe the rate could be higher if we have more personal information at hand. However, this percentage has served its purpose properly. Then We present the top 10 password structures in Table~\ref{t3} and most commonly used personal information in Table~\ref{t4}. Based on Table~\ref{t3} and Table~\ref{t4}, we have the following observations

\begin{table}
\centering
\caption{Most Frequent Password Structures}
\begin{tabular}{|c|c|c|c|} \hline
Rank&Structure&Amount&Percentage\\ \hline
1&D7&7105&5.407\%\\ 
2&[ACCT]&6103&4.644\%\\ 
3&[NAME][BD]&5410&4.117\%\\ 
4&D6&4873&3.708\%\\ 
5&[BD]&4470&3.402\%\\ 
6&D8&4233&3.221\%\\ 
7&L1D7&3286&2.500\%\\ 
8&[NAME]D7&2941&2.238\%\\ 
9&[NAME]D3&2363&1.798\%\\ 
10&[NAME]D6&2241&1.705\%\\ 
\hline\end{tabular}
\label{t3}
\end{table}

\begin{table}
\centering
\caption{Most Popular Personal Information}
\begin{tabular}{|c|c|c|c|} \hline
Rank&Information Type&Amount&Percentage\\ \hline
1&[BD]&30674&23.34\%\\ 
2&[NAME]&29653&22.56\%\\ 
3&[ACCT]&17065&12.98\%\\ 
4&[EMAIL]&4229&3.218\%\\ 
5&[ID]&2918&2.220\%\\ 
6&[CELL]&529&0.402\%\\ 
\hline\end{tabular}
\label{t4}
\end{table}

\begin{enumerate}
\item The second and third structures are perfectly matched to personal information. 3 out of the top 10 structures are composed by pure personal information and 6 out of the top 10 structures have personal information segment. The dominating structures $D_7$, $D_6$, and $D_8$ in Table~\ref{t2} still rank fairly high.
\item Birthday, name, and account name are most popular personal information in user passwords. Over 20\% passwords in our dataset contain birthday or name information. On the other hand, much less people use email and ID number in their passwords. Further more, only few people include their cellphone number in their passwords.  
\item Set aside personal information, digits are still dominating user passwords. Only one structure from the top 10 structures has one letter segment with minimum length (1). The result confirms that Chinese users prefer to use digits in their passwords.
\item An interesting observation is that although account name has merely half percentage as birthday and name information, the structure [ACCT] ranks highest among all structures that contain personal information. The reason behind may be that users tend to use their account names as their passwords instead of using them as part of their passwords. 
\end{enumerate}

\subsubsection{Gender difference}
We are also interested in the difference of password composition between males and females. Note that although the dataset does not have a gender column, user ID number actually has gender information (The second last digit in ID number represents gender). We realized that the dataset is biased in gender, with 9,856 females and 121,533 males in it. To balance the number, we randomly select 9,856 males from the male pool and compare them with females. The average length of passwords for males and females are 8.41 and 8.51, which are quite similar. It shows that males and females do not differ much in the length of their passwords. We then apply the matching method to each of the genders. We found that 55\% of male passwords contain personal information while 45\% of female passwords contain personal information. Therefore we conclude that generally males put more personal information than females in their passwords. It indicates females have wider thought when it comes to password. It also implies that females have more complex passwords, and therefore maybe more secure. We list the top 10 structures for each gender in Table~\ref{t5} and personal information usage in Table~\ref{t6}. From the tables we have the following observations:
\begin{enumerate}
\item For males, 6 out of the top 10 structures contain personal information. Yet for females, only 3 out of top 10 structures contain personal information. It further implies that males are more likely to consider personal information when creating passwords.
\item For both males and females, [ACCT], [NAME][BD], and [BD] are three most frequent structures with personal information. However, The percentage of males are much higher than that of females. Averagely 47.3\% more males are constructing their passwords following the 3 patterns than females.
\item From Table~\ref{t6} we can see the percentage of each type of personal information in the passwords. Interestingly males and females are very different in the usage of name information. Males use their names as frequent as their birthday (23.43\% passwords of males contain their names) while only 13.03\% passwords of females contain names. We also notice that the name usage mostly contribute the 10\% difference in personal information usage between males and females. 
 
\item Users seem not like applying mangling rules on their passwords. We notice that the several most frequent password structures are either pure personal information (such as $[BD]$) or strings (such as $D_6$) that do not relate to any of the personal information. Structures like $[NAME]D_7$ are less likely to appear in user password.

\end{enumerate} 
\begin{table}
\centering
\caption{Most Frequent Structures in Different Gender}
\begin{adjustbox}{max width=0.48\textwidth}
\begin{tabular}{|c|c|c|c|c|} \hline
\multirow{2}{*}{Rank}&\multicolumn{2}{|c|}{Male}&\multicolumn{2}{|c|}{Female}\\ \cline{2-5}
&Structure&Percentage&Structure&Percentage\\ \hline
1 & $D_7$& 5.732\% & $D_6$ & 4.870\%\\ 
2 & [ACCT] & 4.799\% & $D_7$ & 4.220\%\\ 
3 & [NAME][BD] & 4.190\% & [ACCT] & 4.017\%\\ 
4 & [BD] & 3.591\%& $D_8$ & 3.246\%\\ 
5 & $D_6$ & 3.520\% & [NAME][BD] & 2.607\%\\ 
6 & $D_8$ & 2.861\% & [BD] & 2.221\%\\ 
7 & $L_1D_7$ & 2.840\% & $L_2D_6$ & 2.079\%\\ 
8 & [NAME]$D_7$ & 2.333\% & $L_2D_7$ & 1.704\%\\ 
9 & [NAME]$D_6$ & 1.968\% &$L_1D_7$ & 1.684\%\\ 
10 & [NAME]$D_3$ & 1.785\% & $L_3D_6$ & 1.663\%\\
\hline\end{tabular}
\end{adjustbox}
\label{t5}
\end{table}

\begin{table}
\centering
\caption{Most Frequent Personal Information in Different Gender}
\begin{adjustbox}{max width=0.48\textwidth}
\begin{tabular}{|c|c|c|c|c|} \hline
\multirow{2}{*}{Rank}&\multicolumn{2}{|c|}{Male}&\multicolumn{2}{|c|}{Female}\\ \cline{2-5}
&Information Type&Percentage&Information Type&Percentage\\ \hline
1 & [BD] & 23.75\% & [BD] & 20.30\% \\ \hline
2 & [NAME] & 23.43\% & [ACCT] & 13.24\% \\ \hline
3 & [ACCT] & 13.06\% & [NAME] & 13.03\% \\ \hline
4 & [EMAIL] & 2.881\% & [EMAIL] & 4.403\% \\ \hline
5 & [ID] & 2.343\% & [ID] & 1.461\% \\ \hline
6 & [CELL] & 0.263\% & [CELL] & 0.385\% \\ 
\hline\end{tabular}
\end{adjustbox}
\label{t6}
\end{table}


\subsection{Service Information} 
\subsection{Ethical Consideration}
We do realize that studying leaked datasets involves much ethical concern. Like many other works, we only use this dataset for researching purpose. We will not expose any user personal information of in this dataset.
\section{Correlation Quantification}
\subsection{Coverage}

\section{Personal-PCFG, an individual-oriented password cracker}
\subsection{Attack Scenarios}
\section{Password Protection}

\section{Related Workd}
Left blank
\section{REF}






\section{Acknowledgments}

% The following two commands are all you need in the
% initial runs of your .tex file to
% produce the bibliography for the citations in your paper.
\bibliographystyle{abbrv}
\bibliography{sigproc}  % sigproc.bib is the name of the Bibliography in this case
% You must have a proper ".bib" file
%  and remember to run:
% latex bibtex latex latex
% to resolve all references
%
% ACM needs 'a single self-contained file'!
%
%APPENDICES are optional
%\balancecolumns
\appendix
%Appendix A
\section{Full List of Birthday and Name in Matching Algorithm}


\end{document}
