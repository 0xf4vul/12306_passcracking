\\
\section{Personal Info in Passwords}
\label{personalinfo}
Intuitively, people tend to create their passwords based on their
personal information since human beings are limited by their memory --
true random passwords are much harder to remember. We show that user
personal information plays an important role in human-chosen password
generation by dissecting passwords in a mid-sized leaked password
dataset.  Understanding the usage of personal information in passwords
and its security implications can help us to further enhance password
security. To start with, we introduce the dataset used throughout this
study.

%is important for both users and system administrators for it has strong indication to user password security. Users and administrators are expected to enhance their password security by avoiding using easily predictable passwords or patterns.  To start with, we introduce the dataset we used thorough our study. 

\subsection{12306 Dataset}
A number of password datasets have been exposed to the public in
recent years, and they usually contain several thousands to millions
of real passwords. As a result, there are several password measurement
or password cracking works based on analyzing those
datasets~\cite{bonneau2012science, li2014large}. In this paper, a
dataset called 12306 is used to illustrate how personal information is
involved in password creation.

\subsubsection{Introduction to Dataset}
At the end of year 2014, a Chinese dataset is leaked to the public by
anonymous attackers. It is reported that the dataset is collected by
trying passwords from other leaked passwords. We
call this dataset 12306 because all passwords are from a website
www.12306.cn, which is the official website of online railway ticket
reservation system for Chinese users. There is no data available
showing the exact number of users in the 12306 website; however, we
infer at least tens of millions users registered in the system since
it is the only official website for the entire Chinese railway
systems.

The 12306 dataset contains over 130,000 Chinese passwords. Having
witnessed so many large datasets leaked out, the size of the 12306
dataset is just medium. What makes it special is that together with
plaintext passwords, the dataset also includes several types of user
personal information, such as a user's name and the government-issued unique ID number
(similar to the Social Security Number in USA). As the
website requires a real ID number to register and people must provide
real personal information to book a ticket, we consider that information in this dataset pretty reliable.


\subsubsection{Basic Analysis}
We first conduct a simple analysis to reveal some general
characteristics of the 12306 dataset.  For data consistency, we remove
users whose ID number is not 18-digit long. These users may have used
other IDs (e.g., passport number) to register the web system and count
0.2\% of the whole dataset. The dataset contains 131,389 passwords for
analysis after being cleansed. Note that various websites may have
different password creation policies. For instance, with a strict
password policy, users may apply mangling rules (e.g., $abc$ $->$
$@bc$ or $abc1$) to their passwords to fulfill the policy
requirement~\cite{weir2010testing}. Since the 12306 website has
changed its password policy after the password leakage, we do not know
the exact password policy when the dataset was leaked. However, from
the leaked dataset, we infer that the password policy is quite simple
-- all passwords cannot be shorter than 6 symbols. There is no
restriction on what type of symbols are used. Therefore, users are not
required to apply any mangling rule to their passwords.

\begin{table}
\centering
\caption{Most Frequent Passwords.}
\begin{tabular}{|c|c|c|c|} \hline
Rank&Password&Amount&Percentage\\ \hline
1&123456&389&0.296\%\\ 
2&a123456&280&0.213\%\\ 
3&123456a&165&0.125\%\\ 
4&5201314&160&0.121\%\\ 
5&111111&156&0.118\%\\ 
6&woaini1314&134&0.101\%\\ 
7&qq123456&98&0.074\%\\ 
8&123123&97&0.073\%\\ 
9&000000&96&0.073\%\\ 
10&1qaz2wsx&92&0.070\%\\ 
\hline\end{tabular}
\label{t1}
\vspace{-0.1in}
\end{table}


The average length of passwords in the 12306 dataset is 8.44. The most
common passwords in the 12306 dataset are listed in Table~\ref{t1}.
The dominating passwords are trivial passwords (e.g., 123456, a123456,
etc.), keyboard passwords (e.g., 1qaz2wsx, 1q2w3e4r, etc.), and ``I
love you'' type passwords. Both ``5201314" and ``woaini1314" means ``I
love you forever" in Chinese. The most commonly used Chinese passwords
are similar to a previous study~\cite{li2014large}; however, the 12306
dataset is much less congregated. The most popular password ``123456"
counts less than 0.3\% of all passwords while the number is 2.17\%
in~\cite{li2014large}. We believe that the password sparsity is due to
the importance of the website so that users are less prone to use
trivial passwords like ``123456" and there is fewer sybil accounts
because real ID number is needed for registration.

We also study the basic structures of the passwords in 12306. The most
popular password structures are shown in Table~\ref{t2}. Similar to a
previous study~\cite{li2014large}, our result again shows that Chinese
users prefer to use digits in their passwords instead of letters as
English-speaking users. The top five structures all have significant
portion of digits, and at most 2 or 3 letters are appended in
front. We reckon that the reason behind may be Chinese characters are
logogram-based and digits seem to be the best alternative when
creating a password.

\begin{table}
\centering
\begin{threeparttable}
\caption{Most Frequent Password Structures.}
\begin{tabular}{|c|c|c|c|} \hline
Rank&Structure&Amount&Percentage\\ \hline
1&$D_7$&10893&8.290\%\\ 
2&$D_8$&9442&7.186\%\\ 
3&$D_6$&9084&6.913\%\\ 
4&$L_2D_7$&5065&3.854\%\\ 
5&$L_3D_6$&4820&3.668\%\\ 
6&$L_1D_7$&4770&3.630\%\\ 
7&$L_2D_6$&4261&3.243\%\\ 
8&$L_3D_7$&3883&2.955\%\\ 
9&$D_9$&3590&2.732\%\\ 
10&$L_2D_8$&3362&2.558\%\\ 
\hline\end{tabular}
\label{t2}
\begin{tablenotes}
      \small
      \item ``D" represents digits and ``L" represents English letters. The number indicates the segment length. For example, $L_2D_7$ means the password contains 2 letters following by 7 digits.
    \end{tablenotes}
    \end{threeparttable}
\vspace{-0.1in}
\end{table}

Overall, the 12306 dataset is a Chinese password dataset that has
general Chinese password characteristics. However, its passwords are
more sparse than previously studied datasets because users concern
password security more when creating passwords for critical service
systems.

\subsection{Personal Information}
The 12306 dataset not only contains user passwords, but also include
multiple types of personal information listed below.

\begin{verbatim}
1. Name: User's Chinese name.
2. Email address: User's registered email address.
3. Cellphone: User's registered cellphone number.
4. Account name: the username used in the system, 
may contain digits and letters, like "myacct123".
5. ID number: Government issued ID number.
\end{verbatim}

Note the government issued ID number is an 18-digit unique number,
which includes personal information itself. The digits 1-6 represent
the birth place of the owner, the digits 7-14 represent the birthday
of the owner, and the digit 17 represents the gender of the owner --
odd number means male and even number means female. We take out the
8-digit birthday information and treat it separately since birthday is
very important personal information in password creation. Therefore,
we finally have six types of personal information: name, birthday,
email address, cellphone number, account name, and ID number (birthday
excluded).

\subsubsection{New Password Representation}
To better illustrate how personal information correlates to user
passwords, we develop a new representation of password by adding more
semantic symbols besides the conventional ``D", ``L" and ``S" symbols,
which stand for digit, letter, and special symbol, respectively. We
try to match parts of a password to the six types of user personal
information, and express the password with these personal
information. For example, a password ``alice1987abc" can be
represented as $[Name][Birthday]L_3$, instead of $L_3D_4L_3$ as in a
traditional representation. The matched personal information is
denoted by corresponding tags -- [Name] and [Birthday] in this
example; for segments that are not matched, we still use ``D",
``L", and ``S" to describe the symbol types. 

We believe that the representation like $[Name][Birthday]L_3$ is
better than $L_5D_4L_3$ since it more accurately describes the
composition of a user password with more detailed semantic
information. Using this representation, we apply the following
matching method to the entire 12306 dataset to see how these personal
information tags appear in password structures.

\subsubsection{Matching Method}
\label{matchingmethod}
%In order to put personal information in password representations, an essential question will be: How to match the personal information to user passwords? To answer this question, we show the algorithm

We propose a matching method to locate personal information in a
user password, which is shown in Algorithm~\ref{alg1}. The high level idea
is that we first generate all substrings of the password and sort them
in descending length order. Then we match these substrings from the
longest to the shortest to all types of personal information.  If one
match is found, the match function is recursively applied over the
remaining password segments until no further match is found. The
segments that are not matched to any personal information will be then
labeled using the traditional ``LDS" tags.


\begin{algorithm}[h!]
\caption{Personal Information Matching.}
\label{alg1}
\begin{algorithmic}[1]
\Procedure{Match}{$pwd$,$infolist$}
\State $newform \gets$ empty\_string
\If{len($pwd$) == 0}
\State \Return empty\_string
\EndIf
\State $substring \gets$ get\_all\_substring($pwd$)
\State reverse\_length\_sort($substring$)
\For {$eachstring$ \Pisymbol{psy}{206} $substring$}
\If {len($eachstring$) $\ge$ 2}
\If{matchbd($eachstring$,$infolist$)}
\State $tag \gets $ ``[BD]"
\State $leftover \gets pwd$.split($eachstring$)
\State break
\EndIf
\State $\ldots$
\If{matchID($eachstring$,$infolist$)}
\If{tag != None}
\State $tag \gets tag + ``\&[ID]"$
\Else
\State $tag \gets$ ``[ID]"
\EndIf
\State $leftover \gets pwd$.split($eachstring$)
\State break
\EndIf
\Else
\State break
\EndIf
\EndFor
\If{$leftover$.size() $\ge$ 2}
\For{i $\gets$ 0 to $leftover$.size()-2}
\State $newform \gets$ MATCH($leftover[i]$,$infolist$) + $tag$
\EndFor
\State $newform \gets$  MATCH($leftover[leftover.size()-1]$)+$newform$
\Else
\State $newform \gets$ seg($pwd$)
\EndIf
\State $results \gets $extract\_ambiguous\_structures($newform$)
\State \Return $results$
\EndProcedure
\end{algorithmic}
%\vspace{-0.1in}
\end{algorithm}



In Algorithm~\ref{alg1}, we first make sure that the length of a
password segment is at least 2 for matching.  Then we try to match eligible segments to each kind of the personal information. Note that we do not present the specific matching methods
for each type of the personal information (line 10 and line 16) to
keep it clean and simple. Instead we describe the methods as follows.
For the Chinese names, we convert them into Pinyin form, which is
alphabetic representation of Chinese. Then we compare password
segments to 10 possible permutations of a name, such as
lastname+firstname and last\_initial+firstname. If the segment is
exactly same as anyone of the permutations, we consider a match is
found. 
%We list all the 10 permutations in the Appendices. 
For birthday, we list 17 possible permutations and compare a password
segment with these permutations. If the segment is the same as any
permutation, we consider it as a match. 
%All the birthday permutations are also listed in the Appendices. 
For account name, Email address, cellphone number, and ID number, we further restrain
the length of a segment to be at least 3 to avoid mismatching by
coincidence. Besides, as people tend to memorize a sequence of numbers by
dividing it into 3-digit groups, We believe that a match of at least
length 3 is likely to be a real match. 

% If the segment is a substring of any of the 3 personal information, we regard it a match to the corresponding personal information. 
Note that for a password segment, it may match to multiple types of
personal information. In such cases, all matches are counted. The results of Algorithm~\ref{alg1} contain all possible matches. \\

\subsubsection{Matching Results}
\label{matchingresult}
After applying Algorithm~\ref{alg1} to the 12306 dataset, we find that
78,975 out of 131,389 (60.1\%) of the passwords contain at least one
of the six types of personal information. Apparently, personal
information is frequently used in password creation. We believe that
the ratio could be even higher if we have more personal information at
hand. We present the top 10 password structures in Table~\ref{t3} and
the usage of personal information in passwords in Table~\ref{t4}.  As
mentioned above, a password segment may match to multiple types of
personal information and we count all these matches. Therefore, with
131,389 passwords we obtain 153,895 password structures. Based on
Tables~\ref{t3} and~\ref{t4}, we have the following observations.

\begin{table}
\centering
\caption{Most Frequent Password Structures.}
\begin{tabular}{|c|c|c|c|} \hline
Rank&Structure&Amount&Percentage\\ \hline
1&[ACCT]&6820&5.190\%\\
2&D7&6224&4.737\%\\
3&[NAME][BD]&5410&4.117\%\\
4&[BD]&4470&3.402\%\\
5&D6&4326&3.292\%\\
6&[EMAIL]&3807&2.897\%\\
7&D8&3745&2.850\%\\
8&L1D7&2829&2.153\%\\
9&[NAME]D7&2504&1.905\%\\
10&[ACCT][BD]&2191&1.667\%\\
\hline\end{tabular}
\label{t3}
\vspace{-0.1in}
\end{table}

\begin{table}
\centering
\caption{Personal Information Usage.}
\begin{tabular}{|c|c|c|c|} \hline 
Rank & Information Type & Amount & Percentage { \footnote{\small Note
    that one password segment can be matched to multiple types of
    personal information. If so, all matched types are
    counted. Therefore the sum of the percentages is larger than
    60.1\%}}\\ \hline 1&Birthday&31674&24.10\%\\ 2&Account
Name&31017&23.60\%\\ 3&Name&29377&22.35\%\\ 4&Email&16642&12.66\%\\ 5&ID
Number&3937&2.996\%\\ 6&Cell Phone&3582&2.726\%\\ \hline\end{tabular}
\label{t4}
\vspace{-0.1in}
\end{table}


\begin{enumerate}[leftmargin=*]
\item Even for security sensitive websites like 12306, people tend to
  use personal information for password creation. 5 out of the top 10
  structures are composed by pure personal information, and 6 out of
  the top 10 structures have personal information segments.

\item Birthday, account name, and name are the most popular personal
  information contained in user passwords, with 24.10\%, 23.60\%, and
  22.35\% occurrence rates, respectively. Meanwhile, about 12.66\% of
  users include email in their passwords. However, only few percentage
  of people include their cellphone and ID number in their passwords
  (less than 3\%).

\item Digits are still dominating user passwords. The dominating
  structures $D_7$, $D_6$, and $D_8$ in Table~\ref{t2} still rank
  fairly high. Only one structure from the top 10 structures contains
  letter segment. The result confirms that Chinese users prefer to use
  digits in their passwords.

\item %An interesting observation is that although 
%While the number of passwords including account name is clearly lower than those of including birthday and name information, 
The structure [ACCT] ranks the highest among all structures. The
reason may be that users tend to directly use their account names as
their passwords, instead of using them as part of their passwords.
\end{enumerate}

\subsubsection{Gender Password Preference}
\label{genderdifference}
%Beside treating the dataset as a whole, it is also interesting to study the difference of password structures between males and females. Although the dataset does not have a gender column, 

As the user ID number in our dataset actually contains gender
information (i.e., the second last digit in the ID number represents
gender), we compare the password structures between males and females
to see if there is any difference on password preference. Since the
dataset is biased in gender with 9,856 females and 121,533 males, we
randomly select 9,856 males from the male pool and compare them with
females.

The average password lengths for males and females are 8.41 and 8.51,
respectively, which shows that males and females do not differ much in
the length of their passwords. We then apply the matching method to
each gender. We observe that 61.0\% of male passwords contain personal
information while only 54.1\% of female passwords contain personal
information. We list the top 10 structures for each gender in
Table~\ref{t5} and personal information usage in Table~\ref{t6}. These
results demonstrate that male users are more likely to include
personal information in their passwords than female users. In
addition, we have the following two interesting observations:

\begin{enumerate}[leftmargin=*]
%\item 
%From the percentages of passwords containing personal information (54.9\% for males and 44.6\% for females) and the fact that each structure that contained personal information in the top 10 structures has higher percentage for male users, we can easily imply that  Male users are more likely to include personal information in their passwords than female users.

\item As Table~\ref{t5} shows, 28.38\% of males' passwords fall into
  the category of the top 10 structures but there are only 23.94\% of
  female passwords whose structures belong to the top 10. Moreover,
  the total number of different password structures of females is
  1,756, which is 10.3\% more than the total number of males. Thus,
  the passwords of males are more congregated and hence tend to be
  more predictable.

%From Table~\ref{t6} we can see the percentage of each type of personal information in the passwords. 
\item Males and females vary in the usage of name information, as
  shown in Table~\ref{t6}. Males use their names as frequent as their
  birthday (23.31\% passwords of males contain their names). In
  contrast, only 12.94\% of females' passwords contain their names. We
  also notice that the name usage is the dominating factor that causes
  the difference in personal information between males and females. In
  other words, except for name usage, male and female have no significant
  difference in using personal information for password creation.
\end{enumerate} 

\begin{table}
\centering
\caption{Most Frequent Structures in Different Gender.}
\begin{adjustbox}{max width=0.48\textwidth}
\begin{tabular}{|c|c|c|c|c|} \hline
\multirow{2}{*}{Rank}&\multicolumn{2}{|c|}{Male}&\multicolumn{2}{|c|}{Female}\\ \cline{2-5}
&Structure&Percentage&Structure&Percentage\\ \hline
1&[ACCT] & 4.647\%&D6 & 3.909\%\\
2&D7 & 4.325\%&[ACCT] & 3.729\%\\
3&[NAME][BD] & 3.594\%&D7 & 3.172\%\\
4&[BD] & 3.080\%&D8 & 2.453\%\\
5&D6 & 2.645\%&[EMAIL] & 2.372\%\\
6&[EMAIL] & 2.541\%&[NAME][BD] & 2.309\%\\
7&D8 & 2.158\%&[BD] & 1.968\%\\
8&L1D7 & 2.088\%&L2D6 & 1.518\%\\
9&[NAME]D7 & 1.749\%&L1D7 & 1.267\%\\
10&[ACCT][BD] & 1.557\%&L2D7 & 1.240\%\\ \hline
NA&TOTAL&28.384\%&TOTAL&23.937\%\\
\hline\end{tabular}
\end{adjustbox}
\label{t5}
\end{table}

\begin{table}
\centering
\caption{Most Frequent Personal Information in Different Gender.}
\begin{adjustbox}{max width=0.48\textwidth}
\begin{tabular}{|c|c|c|c|c|} \hline
\multirow{2}{*}{Rank}&\multicolumn{2}{|c|}{Male}&\multicolumn{2}{|c|}{Female}\\ \cline{2-5}
&Information Type&Percentage&Information Type&Percentage\\ \hline
1&[BD]&24.56\%&[ACCT]&22.59\%\\
2&[ACCT]&23.70\%&[BD]&20.56\%\\
3&[NAME]&23.31\%&[NAME]&12.94\%\\
4&[EMAIL]&12.10\%&[EMAIL]&13.62\%\\
5&[ID]&2.698\%&[CELL]&2.982\%\\
6&[CELL]&2.506\%&[ID]&2.739\%\\
\hline\end{tabular}
\end{adjustbox}
\label{t6}
\end{table}

In summary, passwords of males are generally composed of more personal
information, especially names of the users. In addition, the password
diversity for males is lower. Our analysis indicates that the
passwords of males are more vulnerable than those
of females. At least from the perspective of personal-information-related attacks, our observations are different from the conclusion drawn in \cite{mazurek2013measuring} that males have slightly stronger passwords than females.
%from the perspective of using personal information in passwords. 


\subsection{Domain Information} 
Cao et al.~\cite{cao2014personalized} proposed an interesting idea
that uses domain information to crack user passwords. It draws our
attention since we have shown the involvement of personal information
in user password and naturally we are also interested in the
involvement of the domain information as another aspect of semantic
information in password creation. By domain information, we mean the
information of an Internet domain, e.g., a web service. For example,
the famous ``Rockyou'' dataset is leaked from a website
$www.rockyou.com$, the domain information here is ``rockyou". In our
personal information study, the domain information is ``12306". It is
reasonable for users to include domain information in their passwords
to keep their passwords different from site to site but still easy to
remember. This approach is promising to balance password security and
memorability; however, the idea has not been validated with a large
scale experiments. Therefore, we attempt to verify whether domain
information is involved in password creation as personal
information. In addition to the medium-sized 12306 dataset, we study
more password datasets including Tianya, Rockyou, PHPBB, and MySpace
datasets. In each dataset, we search the domain information and its meaningful substrings in the
passwords, and the results are shown in Table~\ref{t7}.

\begin{table}[!]
\centering
\caption{Domain Information in Passwords.}
\begin{adjustbox}{max width=0.48\textwidth}
\begin{tabular}{|c|c|c|c|} \hline
Dataset & Password Amount & Domain info Amount & Percentage\\ \hline
Rockyou & 14,344,391 & 44,025 & 0.3\%\\ Tianya & 26,832,592 &
29,430{ \footnote{\small We find that 45,574 passwords in the Tianya
    dataset is ``111222tiany". It does not make much sense for so many
    users using the same password and highly likely they are sybil
    accounts. Thus, these duplications are removed from our analysis.}
} &0.11\%\\ PHPBB & 184,389 & 2,209 & 1.2\%\\ 12306 & 131,389 & 490 &
0.4\%\\ MySpace & 37144 & 72 & 0.2\%\\ \hline\end{tabular}
\end{adjustbox}
\label{t7}
\end{table}

From Table~\ref{t7}, we can see that indeed some users include domain
information in their passwords. Our results indicate that all the
datasets examined contain 0.11\% to 1.2\% of passwords that relate to
domain information. However, the small percentage indicates that
though including domain information in a password helps users to have
different passwords for different websites, currently not many users
are using such a method to construct their passwords.


%\subsection{Ethical Consideration}
%We do realize that studying leaked datasets involves much ethical concern. We claim that we only use the datasets for researching purpose. All data are carefully stored and used. We will not expose any user personal information or password or use these information in any other way except for research use.
