\section{Related Work}

Researchers have done brilliant works on measuring real life
passwords. In one of the earliest work~\cite{morris1979password}, Morris and Thompson found that passwords are quite simple and thus are vulnerable to dictionary attacks. 
%Similar measurement works are usually done on school system passwords
%or publicly available leaked passwords.
Malone et al.~\cite{malone2012investigating} studied the distribution
of passwords on several large leaked datasets and found that user
passwords fit Zipf distribution well. Gaw and Felton~\cite{gaw2006password}
showed how users manage their
passwords. Mazurek et al.~\cite{mazurek2013measuring} measured 25,000
passwords from a university and revealed correlation between
demographic or other factors, such as field of study, and passwords.
Bonneau~\cite{bonneau2012science} studied language effect on user
passwords from over 70 million passwords. Through measuring the
guessability of 4-digit PINs on over 1,100 banking customers
\cite{bonneau2012birthday}, Bonneau et al. found that birthday appears extensively
in 4-digit PINs.
%
Li et al.~\cite{li2014large} conducted a large-scale measurement
study on Chinese passwords, in which over 100 million real life
passwords are studied and differences between Chinese and other
languages passwords are presented.

There are several works investigating specific aspects of
passwords. Yan et al.~\cite{yan2004password} and Kuo et
al.~\cite{kuo2006human} investigated the mnemonic based
passwords. Veras et al.~\cite{veras2012visualizing} showed the
importance of date in passwords. Das et al.~\cite{das2014tangled}
studied how users mangle one password for different sites.
Schweitzer et al.~\cite{schweitzer2009visualizing} studied the
keyboard pattern in passwords. Beside password itself, researches have
been done on human habit and psychology towards password
security~\cite{florencio2007large, howe2012psychology}.

It has been shown that Shannon entropy has lots of troubles accurately
describing the security of passwords~\cite{cachin1997entropy,
  kelley2012guess, pliam2000incomparability,
  weir2010testing}. Researchers developed a number of metrics to
measure passwords. Massey \cite{massey1994guessing} proposed
guessing entropy, which shows the expected number of guesses to make a
correct guess. Several other most commonly used metrics include
marginal guesswork $\mu_\alpha$~\cite{pliam2000incomparability}, which
measures the number of expected guess to succeed with probability
$\alpha$, and marginal success rate
$\lambda_\beta$~\cite{boztas1999entropies}, which is the probability
of succeed in $\beta$ guesses.

Study of password cracking method has been discussed for over three
decades. Attackers usually tried to recover passwords from a hashed
password database. While reverse hashing function is infeasible, early
works found that passwords are vulnerable to dictionary
attacks~\cite{morris1979password}. Time-memory Trade-off in
passwords~\cite{hellman1980cryptanalytic} made dictionary attacks much
more efficient. Rainbow table~\cite{oechslin2003making} reduces table
number in~\cite{hellman1980cryptanalytic} using multiple reduction functions. However, recent years as
the password policy becomes strict, simple dictionary passwords are less
common. Narayanan and Shmatikov~\cite{narayanan2005fast} used
Markov model to generate guesses based on that passwords need to be
phonetically similar to users' native languages. Castelluccia et
al.~\cite{castelluccia2013privacy} improved Markov attack by making
guess in approximately descending probability order. In 2009, Weir
et al.~\cite{weir2009password} leveraged Probabilistic Context-Free
Grammars (PCFG) to crack passwords. Veras et
al.~\cite{veras2014semantic} tried to use semantic patterns in
passwords. Besides, while attacking a hashed password database remains
main attacking scenarios, there are other attacks on different
scenarios, such as video eavesdropping on
passwords~\cite{balzarotti2008clearshot}.

There have been works on protecting passwords by enforcing users to
select more secure passwords, among which password strength meters
seem to be one effective method. Castelluccia et
al~\cite{castelluccia2012adaptive} proposed to use Markov Model as in
~\cite{narayanan2005fast} to measure the security of user
passwords. Meanwhile commercial password meters adopted by popular
websites are proved inconsistent~\cite{de2014very}.  Feedback can be
provided to users using trained leaked passwords or
dictionaries~\cite{weir2010testing, komanduri2014telepathwords}.
Several works discussed adding security questions on top of passwords
~\cite{pinkas2002securing, schechter2009s, brainard2006fourth}. Though
some graphics-based~\cite{davis2004user,jermyn1999design} or
biometrics-based~\cite{jain2006biometrics} authentication methods have
been proposed, text-based passwords are expected to remain
dominating~\cite{bonneau2012quest}.
