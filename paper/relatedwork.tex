\section{Related Work}

Researchers have done brilliant works on measuring real life passwords. One of the earliest work is done by R. Morris and K. Thompson \cite{morris1979password}. They found that passwords are quite simple and thus are vulnerable to dictionary attacks. Similar measurement works are usually done on school system passwords or publicly available leaked passwords. Malone et al \cite{malone2012investigating} studied the distribution of passwords in a statistical way on several large leaked datasets, found that user passwords fit Zipf distribution well. S. Gaw \cite{gaw2006password} shows how users manage their passwords. Mazurek \cite{mazurek2013measuring} measured 25,000 passwords from a university, revealing correlation between demographic or other factors, such as field of study, and passwords.   J. Bonneau \cite{bonneau2012science} studied language effect on user passwords from over 70 million passwords. He also measured the guessability of 4-digit PINs on over 1,100 banking customers \cite{bonneau2012birthday}, finding that birthday appears extensively in 4-digit PINs. Z. Li et al \cite{li2014large} conducted a large-scale measurement study on Chinese passwords, in which over 100 million real life passwords are studied and differences between Chinese and other languages passwords are presented. There are several works investigating specific aspects of passwords. J. Yan \cite{yan2004password} and C. Kuo \cite{kuo2006human} investigated the mnemonic based passwords. R.  \cite{veras2012visualizing} showed the importance of date in passwords. A. Das et al \cite{das2014tangled} studied how users mangle one password for different sites. And D. Schweitzer \cite{schweitzer2009visualizing} studied the keyboard pattern in passwords.  Beside password itself, researches have been done on human habit and psychology towards password security \cite{florencio2007large}\cite{howe2012psychology}. 

It has been shown that Shannon entropy has lots of trouble accurately describing the security of passwords \cite{cachin1997entropy}\cite{kelley2012guess}\cite{pliam2000incomparability}\cite{weir2010testing}. Researchers developed many metrics to measure passwords. J Massey \cite{massey1994guessing} proposed guessing entropy, which shows the expected number of guesses to make a correct guess. Several other most commonly used metrics include marginal guesswork $\mu_\alpha$ by J. Bonneau \cite{pliam2000incomparability}, which measures the number of expected guess to succeed with probability $\alpha$,  marginal success rate $\lambda_\beta$ by Boztas \cite{boztas1999entropies}, which is the probability of succeed in $\beta$ guesses.

Study of password cracking method has been discussed for over three decades. Attackers usually tried to recover passwords from a hashed password database. While reverse hashing function is infeasible, early works found that passwords are vulnerable to dictionary attacks \cite{morris1979password}. Time-memory Trade-off in passwords proposed by M.Hellman \cite{hellman1980cryptanalytic} made dictionary attacks much more efficient. Rainbow table proposed by P.Oechslin \cite{oechslin2003making} is derived from Hellman method \cite{hellman1980cryptanalytic} by reducing table number while use multiple reduction functions. However, recent years as the password policy becomes strict, simple dictionary password is less common. A. Narayanan \cite{narayanan2005fast} used Markov model to generate guesses based on that passwords need to be phonetically similar to users' native languages. M.Durmuth \cite{castelluccia2013privacy} improved Markov attack by making guess in approximately descending probability order. In 2009, M. Weir \cite{weir2009password} leverages Probabilistic Context-Free Grammars (PCFG) to crack passwords. R. Veras et al\cite{veras2014semantic} tried to use semantic patterns in passwords, their method can be seen as a revolution of PCFG by M. Weir. Besides, while attacking a hashed password database remains main attacking scenarios, there are also other attacks on different scenarios, such as video eavesdropping on passwords done by D. Balzarotti \cite{balzarotti2008clearshot}.

There have been works discussing protecting passwords by enforcing users to select more secure passwords, among which password strength meters seem to be one effective method. C. Castelluccia et al \cite{castelluccia2012adaptive} proposed to use Markov Model as in \cite{narayanan2005fast} to measure the security of user passwords. Meanwhile commercial password meters adopted by popular websites are proved inconsistent \cite{de2014very}. Both M. Weir \cite{weir2010testing} and S. Komanduri \cite{komanduri2014telepathwords} tried to provide feedback to users using trained leaked passwords or dictionaries. Several works discussed adding security questions on top of passwords \cite{pinkas2002securing}\cite{schechter2009s}\cite{brainard2006fourth}. Some works propose to use graphics \cite{davis2004user}\cite{jermyn1999design} or biometrics \cite{jain2006biometrics} instead of passwords for authentication. However, text-based passwords are expected to remain dominating \cite{bonneau2012quest}.