\begin{abstract}
While it is not recommended, Internet users often include parts of personal information in their passwords for easy memorization. However, the use of personal information in passwords and its security implications have not yet been studied systematically in the past. 
In this paper, we first dissect user passwords from a leaked dataset to investigate how and to what extent user personal information resides in a password. In particular, we extract the most popular password structures expressed by personal information, and we introduce a new metric called coverage to quantify the correlation between passwords and personal information. Then, exploiting the potential of cracking passwords based on our analysis, we develop a semantics-richer Probabilistic Context-Free Grammars method called Personal-PCFG to crack passwords. Through offlline and online attack scenarios, we demonstrate that Personal-PCFG cracks passwords much faster than the state-of-art techniques and makes online attacks much easier to succeed.
To defend against such semantics-aware attacks, we propose to use distortion functions that are chosen by users to mitigate unwanted correlation between personal information and passwords. Our experimental results show that a simple distortion can effectively protect a password from personalized cracking without sacraficing usability.
\end{abstract}

% A category with the (minimum) three required fields
\category{}{Security and privacy}{}[Human and societal aspects of security and privacy]
%A category including the fourth, optional field follows...
\category{}{General and reference}{}[Metrics]

\terms{Security}

\keywords{passwords, password cracking, data processing, password protection}
